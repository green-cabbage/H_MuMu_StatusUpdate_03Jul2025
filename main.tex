\documentclass[dvipsnames,aspectratio=169]{beamer}


% Setup appearance:
%\usepackage[usenames,dvipsnames,svgnames,table]{xcolor}
\usetheme{metropolis}

\usepackage[]{xcolor}
\usepackage{hyperref}
\usepackage[export]{adjustbox}

%\usetheme{Darmstadt}
%\mode<presentation>
%{
%  %\usetheme{Copenhagen}
%  \usetheme{Dresden}
%  %\usetheme{Darmstadt}
%  \useoutertheme{smoothbars}
%  \useinnertheme{rounded}
%  \usecolortheme{orchid}
%  \usecolortheme{whale}

%  \useoutertheme{infolines}

%  \setbeamerfont{block title}{size={}}

%  \setbeamercovered{transparent}

%  \definecolor{darkred}{RGB}{170,20,20}
%  \definecolor{jg-gray}{RGB}{100,100,100}
%  \setbeamercolor*{structure}{fg=darkred}
%  %\setbeamercolor*{structure}{fg=jg-gray}

%  \usefonttheme{professionalfonts}
%}
%\usefonttheme[onlylarge]{structurebold}
\setbeamerfont*{frametitle}{size=\large,series=\bfseries}
\setbeamertemplate{frametitle}[default][center]
%\setbeamerfont*{textnormal}{size=\normalsize}
\setbeamertemplate{navigation symbols}{}
\setbeamercovered{transparent}




%\setbeamercolor{normal text}{fg=black,bg=white}
%\setbeamercolor{structure}{fg=black}




%\setbeamercolor{block body}{use=block title,bg=white}% Standard packages
\usepackage[english]{babel}
\usepackage[utf8]{inputenc}
%\usepackage{times}
%\usepackage[T1]{fontenc}
\usepackage{caption}
%\captionsetup{figurename=Fig.}
\usepackage{multirow}
\usepackage{ulem}
%\usepackage{mathpazo}
%\usepackage{feynmp}
\unitlength = 1mm
\usepackage{amsmath}
\usepackage{amssymb}
\usepackage{overpic}
\usepackage{rotating}
% Setup TikZ
\usepackage{appendixnumberbeamer}
\usepackage{hyperref}
\newenvironment{rcases}{%
  \left.\renewcommand*\lbrace.%
  \begin{cases}}%
{\end{cases}\right\rbrace}

\usepackage{tikz}

 
\makeatletter
\def\tikzonnode#1{%
\pgfutil@ifnextchar[{\tikzonnode@opt#1}{\tikzonnode@opt#1[]}%
}
\def\endtikzonnode{%
\end{scope}%
}
\def\tikzonnode@opt#1[#2]{%
\pgfpointanchor{#1}{south west}%
\pgfgetlastxy\tse@tikz@shift@x\tse@tikz@shift@y
\begin{scope}[
shift={(\tse@tikz@shift@x,\tse@tikz@shift@y)},
x={(#1.south east)},
y={(#1.north west)},
#2]%
}
 
\def\tikzonimage{%
\pgfutil@ifnextchar[{\tikzonimage@opt}{\tikzonimage@opt[]}%
}
\def\tikzonimage@opt[#1]#2{%
\begin{tikzpicture}
\node[inner sep=0] (image) {\includegraphics[#1]{#2}};
\begin{tikzonnode}{image}%
}
\def\endtikzonimage{%
\end{tikzonnode}
\end{tikzpicture}%
}
\makeatother

\newcommand{\textStamp}[1]{\Large\color{red}\fbox{#1}\normalsize\color{black}}


\newcommand{\overlayGraphic}[4]{
        % graphic option, path, position, text
        \begin{tikzonimage}[#1]{#2}
                \path (#3) node {#4};
        \end{tikzonimage}
}

\usepackage{graphicx}
\usepackage{xspace}
\usepackage{multirow}
\usepackage{subfigure}
\usepackage[]{hyperref}
\definecolor{darkblue}{rgb}{0,0,.5}
\hypersetup{pdftex=true, colorlinks=true, breaklinks=true, linkcolor=darkblue, menucolor=darkblue, pagecolor=darkblue, urlcolor=darkblue,urlbordercolor={1 0.5 0.25}}
\definecolor{ao}{rgb}{0.0, 0.5, 0.0}
\definecolor{darkgreen}{rgb}{0.0, 0.5, 0.0}
\definecolor{byzantine}{rgb}{0.74, 0.2, 0.64}
%\usepackage{ptdr-definitions}
%\usetikzlibrary{arrows}
%\tikzstyle{block}=[draw opacity=0.7,line width=1.4cm]
% \newcommand{\NMC}{\ensuremath{N_{MC}}\xspace}
% \newcommand{\NSM}{\ensuremath{N_{SM}}\xspace}
% \newcommand{\pLN}{\ensuremath{p_{LN}}\xspace}
% \newcommand{\pNorm}{\ensuremath{p_{Norm}}\xspace}
% \newcommand{\mustar}{\ensuremath{\mu^*}\xspace}
% \newcommand{\fstar}{\ensuremath{f^*}\xspace}
% \newcommand{\mstar}{\ensuremath{m^*}\xspace}
% \newcommand{\Ptil}{\ensuremath{\tilde{P}}\xspace}
% 
% % Measurements and units...
% 
% \newcommand{\de}{\ensuremath{^\circ}}
% \newcommand{\ten}[1]{\ensuremath{\times \textnormal{10}^\textnormal{#1}}}
% \newcommand{\unit}[1]{\ensuremath{\textnormal{\,#1}}\xspace}
% \newcommand{\mum}{\ensuremath{\,\mu\textnormal{m}}\xspace}
% \newcommand{\micron}{\ensuremath{\,\mu\textnormal{m}}\xspace}
% \newcommand{\cm}{\ensuremath{\,\textnormal{cm}}\xspace}
% \newcommand{\mm}{\ensuremath{\,\textnormal{mm}}\xspace}
% \newcommand{\fm}{\ensuremath{\,\textnormal{fm}}\xspace}
% \newcommand{\ns}{\ensuremath{\,\textnormal{ns}}\xspace}
% \newcommand{\mus}{\ensuremath{\,\mu\textnormal{s}}\xspace}
% \newcommand{\keV}{\ensuremath{\,\textnormal{ke\hspace{-.08em}V}}\xspace}
% \newcommand{\MeV}{\ensuremath{\,\textnormal{Me\hspace{-.08em}V}}\xspace}
% \newcommand{\GeV}{\ensuremath{\,\textnormal{Ge\hspace{-.08em}V}}\xspace}
% \newcommand{\geV}{\ensuremath{\,\textnormal{Ge\hspace{-.08em}V}}}
% \newcommand{\TeV}{\ensuremath{\,\textnormal{Te\hspace{-.08em}V}}\xspace}
% \newcommand{\teV}{\ensuremath{\mathrm{\; TeV}}\xspace}
% \newcommand{\PeV}{\ensuremath{\,\textnormal{Pe\hspace{-.08em}V}}\xspace}
% 
% \newcommand{\pbinv} {\mbox{\ensuremath{\,\textnormal{pb}^\textnormal{\scriptsize{$-$1}}}}\xspace}
% \newcommand{$fb^{-1}} {\mbox{\ensuremath{\,\textnormal{fb}^\textnormal{\scriptsize{$-$1}}}}\xspace}
% \newcommand{\nbinv} {\mbox{\ensuremath{\,\textnormal{nb}^\textnormal{$-$1}}}\xspace}
% \newcommand{\percms}{\ensuremath{\,\textnormal{cm}^\textnormal{$-$2}\,\textnormal{s}^\textnormal{$-$1}}\xspace}
% \newcommand{\lumi}{\ensuremath{\mathcal{L}}}
% \newcommand{\Lumi}{\ensuremath{\mathcal{L}}\xspace}
% 

\usepackage{ulem}
\usepackage{hyperref}
\newcommand{\EE}{\ensuremath{e^{+}e^{-}}\xspace}
\newcommand{\GeV}{~GeV\xspace}
\newcommand{\mll}{\ensuremath{m_{ll}}\xspace}
\newcommand{\MM}{\ensuremath{\mu^{+}\mu^{-}}\xspace}
\newcommand{\PW}{W}
\newcommand{\Z}{Z}
\newcommand{\Ztautau}{\ensuremath{\Z\to\tau\tau}}
\newcommand{\cls}{\ensuremath{\mathrm{CL_S}}}
\newcommand{\tauh}{\ensuremath{\tau_\text{h}}}
\newcommand{\rmue}{\ensuremath{r_{\Pgm{}\Pe}}\xspace}
\newcommand{\nvert}{\ensuremath{N_\text{Vertices}}}
\newcommand{\njets}{\ensuremath{N_\text{Jets}}}
\newcommand{\mcut}{\ensuremath{m_\text{max}}}
\newcommand{\eps}{\ensuremath{\epsilon}}
\newcommand{\CLS}{\cls}
\newcommand{\Rinout}{\ensuremath{R_\text{in/out}}\xspace}
\newcommand{\Rsfof}{\ensuremath{R_\text{SF/OF}}\xspace}
\newcommand{\etalep}{\ensuremath{\eta_\text{lep}}\xspace}

\newcommand{\sSeven}{\ensuremath{\sqrt{s}=7\TeV}\xspace}
\newcommand{\sEight}{\ensuremath{\sqrt{s}=8\TeV}\xspace}
\newcommand{\lepSec}{The applied lepton selection changed in the \sEight-MC with respect to the \sSeven-MC}

%\newcommand{\eepm}{\Pep\Pem}
%\newcommand{\mmpm}{\Pgmp\Pgmm}
\newcommand{\EM}{\ensuremath{e^{\pm} \mu^\mp}}

%PAS things
\newcommand{\Ht}{\HT}
\newcommand{\Njets}{$N_{\text{jets}}$\xspace}
\newcommand{\MET}{$E_{\mathrm{T}}^{\mathrm{miss}}$\xspace}
\newcommand{\HT}{$H_{\mathrm{T}}$\xspace}
\newcommand{\pt}{$p_{\mathrm{T}}$\xspace}
\newcommand{\wjets}{\ensuremath{\PW+\text{jets}}\xspace}
\newcommand{\zjets}{\ensuremath{\Z+\text{jets}}\xspace}
\newcommand{\DYjets}{\ensuremath{\text{DY}+\text{jets}}\xspace}
\newcommand{\ttll}{$\ttbar\rightarrowl^+l^-$}
\newcommand{\tttau}{$\ttbar\rightarrow\tau^+l^-$}
\newcommand{\tthad}{$\ttbar\rightarrow{}q\bar{q}$}
\newcommand{\ttbar}{$t\bar{t}$}
\newcommand{\lumifinal}{9.2\fbinv}
\newcommand{\fbinv}{$fb^{-1}$}
%\newcommand{\lumi}{\ensuremath{0.920 \; \fbi}}


\newcommand{\secondchi}{\ensuremath{{\tilde{\chi}^0_2}}\xspace}
\newcommand{\firstchi}{\ensuremath{{\tilde{\chi}^0_1}}\xspace}
\newcommand{\sbottom}{\ensuremath{\tilde{b}}\xspace}
\newcommand{\slepton}{\ensuremath{\tilde{l}}\xspace}

\newcommand{\JZB}{\ensuremath{\mathrm{JZB}}\xspace}
\newcommand{\jzb}{\JZB}
\newcommand{\ptll} {\ensuremath{\pt_{ll}}}
\newcommand{\CTEQ} {{\textsc{cteq}}}

\newcommand{\central}{central\xspace}
\newcommand{\inclusive}{inclusive\xspace}
\newcommand{\cfAN}[1]{(cf.~#1)}
\newcommand{\ethAN}[1]{\cite{JZBAN} Sec.~#1}
\newcommand{\ethANApp}[1]{\cite{JZBAN} App.~#1}
\newcommand{\acAN}[1]{\cite{EdgeAN} Sec.~#1}
\newcommand{\acANApp}[1]{\cite{EdgeAN} App.~#1}
\newcommand{\benAN}[1]{\cite{TemplatesAN} Sec.~#1}
%\newcommand{\ethAN}[1]{\texttt{AN-12-231} sec. #1}
%\newcommand{\acAN}[1]{\texttt{AN-12-200} sec. #1}
%\newcommand{\benAN}[1]{\texttt{AN-12-359} sec. #1}
\newcommand{\addAN}[1]{\fixme{#1}}
% % physics symbols
% \newcommand{$E_T^{miss}$}{\ensuremath{{E\!\!\!/}_{\mathrm{{T}}}}\xspace}
% \newcommand{\MINV}{\ensuremath{M_{inv}}\xspace}
% %\newcommand{\Wprime}{\ensuremath{\mathrm{W' \rightarrow \mu + \nu}}\xspace}
% \def\Wprime{$W^\prime$\ }
%\def\pt{$p_{T}$\ }
% \def\MT{$M_{T}$\ }
% \def\ttbar{$t\bar{t}$\ }
% \def\WprimeENu{$W^\prime \rightarrow e \nu $\ }
% Author, Title, etc.
% \AtBeginSection[]
% {
%  \begin{frame}
%       \frametitle{\"Ubersicht}
%       \tableofcontents[currentsection]
%   \end{frame}
% }
%\setbeamerfont{section title}{%
 % family=\rmfamily,series=\bfseries,size=\normalsize}
\setbeamercolor{section title}{bg=white,fg=white}
\setbeamertemplate{section page}
{
  \begin{centering}
    {\usebeamerfont{section name}\usebeamercolor[fg]{section name}}
    \vskip1em\par
    %\begin{beamercolorbox}[sep=16pt,center]{part title}
      \usebeamerfont{section title}\insertsection\par
    %\end{beamercolorbox}
  \end{centering}
}
\def\sectionpage{\usebeamertemplate*{section page}}
%\AtBeginSection{\frame{\sectionpage}}



% \titlegraphic{\begin{columns}
% \column{0.5\textwidth}
% \centering
% % \includegraphics[width=0.75\textwidth]{images/Purdue-Sig-Black-Gold-rgb.png}
%  \end{columns} 	}


 \titlegraphic{\includegraphics[width=4.2cm]{images/Purdue-Sig-Black-Gold-rgb.png}}
% 
\title[]{\huge{H $\rightarrow$ $\mu\mu$ analysis Status Update}} 

 \author[Hyeon-Seo Yun]{Hyeon-Seo Yun, Ram Krishna Sharma, Norbert Neumeister}


 \date[\today]{\today}

\institute[Purdue University]{}


\renewcommand{\footnoterule}{\vspace{2ex} \hrule width 0.3\textwidth height 0.2pt \vspace{1ex}} % lowers the footnote position

\begin{document}


\begin{frame}
  \titlepage
\end{frame}


\begin{frame}
    \frametitle{Interest \& Focus of Purdue Group}
    \begin{itemize}
    \item Our Interest
    \begin{itemize}
    \item Work on a list of improvements contributing to the analysis (shown later in this presentation), which we validate on Run~2 samples.
    \item Contribute in both ggH + VBF channels for Run~2 \& Run~3 $H\rightarrow \mu\mu$ analysis. 
    \item Contribute nanoAOD-based framework workflow, which we validate on Run2 and improved over the years.
    \end{itemize}
    \item Focus
    \begin{itemize}
    \item Overall goal: combine Run~2 + Run~3 to reach to 5$\sigma$.
    \item Target: Improve analysis to achieve a significance on the order of 5\% better than in Run~2~\footnote{https://indico.cern.ch/event/1451198/#sc-26-2-contribution-2} to reach overall 5$\sigma$.
    \item Current focus: achieve 5\% improvement in significance primarily from improvement in $m_{\mu\mu}$ resolution, production channel categorization and modeling and test this with Run~2 data.
    \item Afterwards, apply our improvement to 2022 \& 2023 samples.
    \end{itemize}
    \end{itemize}
\end{frame}

\begin{frame}
    \frametitle{Projection of $m_{\mu\mu}$ resolution impact on significance}
    \begin{itemize}
    \item A previous study done by Dmitry Kondratyev estimated 10\% improvement in $m_{\mu\mu}$ resolution for 5\% improvement in significance (AN-2022/145).
    \item A separate study estimates 7\% improvement in $m_{\mu\mu}$ resolution is needed to improve significance by 5\% in ggH channel.
    \item Overall, if we aim for a 5\% improvement in significance purely from improvement in $m_{\mu\mu}$ resolution, we estimate 7-10\% $m_{\mu\mu}$ resolution improvement is necessary.
    \end{itemize}
    \begin{figure}[!h]
    \centering
    {\includegraphics[width=0.6\textwidth]{images/projection/projection_dmitry.png}}
    \end{figure}
\end{frame}

\begin{frame}
    \frametitle{Run~2 re-analysis}
    \begin{itemize}
    \item Use ultra legacy (UL) samples instead of RERECO campaign.
    \begin{itemize}
    \item Better rejection of bad quality events (about $0.01\%$ less events).
    \item Better muon alignment that in turn improves $m_{\mu\mu}$ resolution.
    \item Improved jet energy resolution and correction.
    \end{itemize}
    \item Not all MC from previous iteration are available for UL
    \begin{itemize}
    \item Study MiNNLO DY samples as potential replacement of mass binned DY sample.
    \item VBF-filtered DY not available for UL: Private production of 72 million events.
    \item ggH AMC@NLO sample not available for UL.
    \end{itemize}
    \item Improve $m_{\mu\mu}$ resolution by applying beamspot constraint (BSC) fit.
    \begin{itemize}
    \item Produced nanoAODv12 UL samples as nanoAODv9 doesn't contain variables necessary for BSC fit.
    \end{itemize}
    \item New nanoAOD-based framework was developed $\Rightarrow$ Faster and Cleaner.
    \item Compared dimuon kinematics of different signal MC samples
    \item Z $p_\mathrm{T}$ correction was re-developed with a polynomial order obtained from F-test.
    \item New XGBoost based BDT was trained and used in our UL workflow.
    \end{itemize}
\end{frame}

% \begin{frame}
%     \frametitle{Run2 Re-analysis p2}
%     \begin{itemize}
%     \item Improve $m_{\mu\mu}$ resolution by applying beamspot constraint (BSC) fit
%     \begin{itemize}
%     \item Produced nanoAODv12 UL samples as nanoAODv9 doesn't contain variables necessary for BSC fit
%     \end{itemize}
%     \item New NanoAOD-based framework was developed $\Rightarrow$ Faster and Cleaner
%     \item Zpt correction was re-developed with a polynomial order obtained from F-test
%     \item New XGBoost based BDT was trained and used in our UL workflow
%     \end{itemize}
% \end{frame}

\begin{frame}
    \frametitle{Highlight: beamspot constraint fit}
    % {
    % \fontsize{11pt}{11pt}\selectfont 
    \begin{itemize}
    \item In the past, our group worked with muon POG on the method to improve muon $p_\mathrm{T}$ resolution: \textbf{beamspot constraint (BSC) fit} \footnote{https://indico.cern.ch/event/1336377/#1-vxbs-method-in-run-3-and-nan}
    \begin{itemize}
        \item This improves $p_\mathrm{T}^\mu$ resolution, which in turn improves $m_{\mu\mu}$ resolution.
        \item Successfully merged the BS calibrated variables in CMSSW
        \item New variables are available in nanoAODv12 or newer
    \end{itemize} 
    \item We apply BSC fit last after applying all other corrections (i.e. Rochester, FSR). 
    
    \end{itemize}
    % }
        \begin{columns}
    \begin{column}{0.6\textwidth}
    {
    % \fontsize{9pt}{9pt}\selectfont 
    \begin{itemize}
    \item We observe $\approx6\%$ improvement in $m_{\mu\mu}$ resolution vs Geofit
    \item We propose to make BSC as default correction replacing Geofit.
    \begin{itemize}
    \item Other Higgs analysis groups (ie $H\rightarrow ZZ$) are migrating to BSC fit.
    \end{itemize}
    \end{itemize}
    }
    \end{column}

    \begin{column}{0.4\textwidth}
            \begin{figure}
        \centering
        \includegraphics[width=5.5cm, height=4cm]{images/BSC/BSC_geofit_comparison.pdf}
    \end{figure}
    \end{column}

    \end{columns}
\end{frame}


\begin{frame}
    \frametitle{Highlight: event-by-event $m_{\mu\mu}$ resolution calibration}
    \begin{columns}
        
    \begin{column}{0.5\textwidth}
    
    \begin{itemize}
    \item { In ggH channel, BDT is trained with weight of inverse event-by-event $m_{\mu\mu}$ resolution applied.}
    \item { Therefore, an accurate value of event-by-event $m_{\mu\mu}$ resolution is needed, and new calibration factors were developed.}
    \item {Closure test is performed for validation, and results show improved event-by-event $m_{\mu\mu}$ resolution.}
    \item To access the calibration factors, please contact us!
    \end{itemize}
    \end{column}

    \begin{column}{0.5\textwidth}
            \begin{figure}
        \centering
        \includegraphics[width=\linewidth]{images/closure_2018C.png}%
        % \caption{Caption}
        \label{fig:enter-label}
    \end{figure}
    \end{column}

    \end{columns}
\end{frame}



\begin{frame}
    \frametitle{Highlight: new nanoAOD based framework}
    \begin{itemize}
    \item Originally designed and developed by Dmitry Kondratyev for previous iteration of this analysis.
    \item Framework was updated with latest coffea package
    \begin{itemize}
    \item Awkward-array and HEP vector for jagged array compute + Dask for parallelization
    \end{itemize}
    \item Implemented on Purdue Analysis Facility \footnote{https://analysis-facility.physics.purdue.edu/en/latest/}
    \item Yaml-file based configuration for clean user interface (based on PocketCoffea)
    \item Dask Gateway is used to scale-out workflow \& automate worker allocation (Max workers: $\approx200$)
    \item Performance for full Run~2 Data \& MC:
    \begin{itemize}
    \item Runtime: 6 hours
    \item Output storage: 304 GB
    \item Scale out CPU cores: 160
    \end{itemize}
    \item \href{https://github.com/green-cabbage/copperheadV2}{Github link to framework}
    \end{itemize}
\end{frame}

\begin{frame}
    \frametitle{Highlight: ggH signal MC $m_{\mu\mu}$ resolution from muon alignment}
    \begin{itemize}
    \item Using official nanoAODv9 samples (with no improvement), we look at possible improvement at $m_{\mu\mu}$ resolution due to UL transition
    \item ggH signal RERECO \& UL samples were divided into muon detector regions of leading and sub-leading muon categories (muon $|\eta|$ category)
    \item Each category was fit with a double crystal ball in $110\leq m_{\mu\mu} \leq 150$ GeV region and the fit $m_{\mu\mu}$ resolution ($\sigma$) was extracted
    \end{itemize}
    \begin{table}[h]
    \centering
    \begin{tabular}{lccc}
        \hline
        Muon $|\eta|$ Category & RERECO fit $\sigma$ & UL fit $\sigma$ & Improvement \\
        \hline
        OE & $2.35 \pm 0.01$ & $2.31 \pm 0.02$ & 1.86\% \\
        \hline
    \end{tabular}
\end{table}
\begin{itemize}
\item We observe a small overall improvement in $m_{\mu\mu}$ resolution, and significant improvement when leading and sub-leading muons are detected in overlap-endcap (OE) regions of the detector.
\end{itemize}
\end{frame}

% \begin{frame}
%     \frametitle{Highlight: different ggH signal MC modeling}
%     \begin{itemize}
%     \item In ggH channel, simulated signal distributions of $m_{\mu\mu}$ $p_T^{\mu\mu}$ significantly impact our results
%     \begin{itemize}
%     \item $m_{\mu\mu}$ resolution has a direct impact in combine fit
%     \item $p_T^{\mu\mu}$ is the most important input feature in ggH channel BDT
%     \end{itemize}
%     \item In previous iteration, ggH powheg was used to train BDT but ggH AMC@NLO was used for ultimate evaluation
%     \item We compared the distributions of $m_{\mu\mu}$, $p_T^{\mu\mu}$ between ggH powheg and ggH AMC@NLO
%     \item We observe significant difference in $p_T^{\mu\mu}$ distribution (RIGHT) between powheg and AMC@NLO
%     \item This suggests that we need to use the same sample for train \& evaluation of ggH channel BDT.
%     % \item This leads to simulated $m_{\mu\mu}$ resolution difference for powheg sample $\approx1.4 \%$
    
%     \end{itemize}
% \begin{figure}[!h]
% \centering
% {\includegraphics[width=0.32\textwidth]{images/ggH_sampleStudy/RerecMcModelComparison_mu1_pt.pdf}}
% {\includegraphics[width=0.32\textwidth]{images/ggH_sampleStudy/RerecMcModelComparison_mu2_pt.pdf}}
% {\includegraphics[width=0.32\textwidth]{images/ggH_sampleStudy/RerecMcModelComparison_dimuon_pt.pdf}}
% \end{figure}
% \end{frame}



\begin{frame}{Highlight: different ggH signal MC modeling}
    \begin{columns}
        % Left column for text (takes 60% width)
        \begin{column}{0.7\textwidth}
        {\fontsize{10pt}{12pt}\selectfont 
            \begin{itemize}
            \item In ggH channel, simulated signal distributions of $m_{\mu\mu}$ and $p_\mathrm{T}^{\mu\mu}$ significantly impact our results
            \begin{itemize}
            \item $m_{\mu\mu}$ resolution has a direct impact in combine fit.
            \item $p_\mathrm{T}^{\mu\mu}$ is the most important input feature in ggH BDT.
            \end{itemize}
            \item In previous iteration, ggH Powheg was used to train BDT but ggH AMC@NLO was used for ultimate evaluation.
            \item We compared the distributions of $m_{\mu\mu}$, $p_\mathrm{T}^{\mu\mu}$ between ggH Powheg and ggH AMC@NLO.
            \item We observe similarity in $m_{\mu\mu}$ (top), but significant difference in $p_\mathrm{T}^{\mu\mu}$ distribution (bottom) between Powheg and AMC@NLO.
            \item Results suggest that we need to use consistent ggH MC for train \& evaluation.
            \end{itemize}
        }
        \end{column}
        
        % Right column for figure (takes 40% width)
        \begin{column}{0.3\textwidth}
            \centering
            {\includegraphics[width=0.9\textwidth]{images/ggH_sampleStudy/RerecMcModelComparison_dimuon_mass.pdf}}\\
            {\includegraphics[width=0.9\textwidth]{images/ggH_sampleStudy/RerecMcModelComparison_dimuon_pt.pdf}}
        \end{column}
    \end{columns}
\end{frame}

% \begin{frame}
%     \frametitle{Highlight: simpler Zpt correction}
%     \begin{itemize}
%     \item We determine the maximum order of polynomial fit to apply to our correction using F-test. Result: maximum five (previously 12)
%     \item We apply polynomial fit of Data/ (DY MC) over $p^{\mu\mu}_T$ from Z boson control region divided by jet multiplicity category and obtain Zpt correction SF as function of $p^{\mu\mu}_T$
%     \end{itemize}
% \begin{figure}[!h]
% \centering
% {\includegraphics[width=0.32\textwidth]{images/zpt/dimuon_pt_no_zpt.pdf}}
% {\includegraphics[width=0.32\textwidth]{images/zpt/dimuon_pt_yes_zpt.pdf}}
% \end{figure}
% \end{frame}

\begin{frame}{Highlight: Z $p_\mathrm{T}$ correction}
    \begin{columns}
        % Left column for text (takes 60% width)
        \begin{column}{0.75\textwidth}
            {\fontsize{10pt}{12pt}\selectfont 
            \begin{itemize}
            \item Due to missing resummation QCD effect in DY simulation, significant data/MC disagreement is observed at low $p_\mathrm{T}^{\mu\mu}$ region.
            \item In the previous iteration, 12th order polynomial fit was applied as a function of $p_\mathrm{T}^{\mu\mu}$ within Z control region.
            \item Separate fits are done for each era and different NJet multiplicities.
            \item We ran F-test to obtain order of polynomial fit, which lead to maximum order of five.
            % \item We determine the maximum order of polynomial fit to apply to our correction using F-test.
            % \begin{itemize}
            % Result: maximum order of five (previously 12)
            % \end{itemize}
            % \item We apply polynomial fit of Data/ (DY MC) over $p^{\mu\mu}_T$ from Z boson control region divided by jet multiplicity category and obtain Zpt correction SF as function of $p^{\mu\mu}_T$
            \item Overall, we observe improvements in data/MC compared to previous iteration (i.e. $\approx$10 \% improvement for 2018 Njet=1)
            \item Z $p_\mathrm{T}$ correction applied $p_\mathrm{T}^{\mu\mu}$ in Higgs mass region from HIG-19-006 (Top) and our results (Bottom) for 2018 Njet=1 category
            \end{itemize}
            }
        \end{column}
        
        % Right column for figure (takes 40% width)
        \begin{column}{0.25\textwidth}
            \centering
            % {\includegraphics[width=0.9\textwidth]{images/zpt/dimuon_pt_no_zpt.pdf}}\\
            % {\includegraphics[width=0.9\textwidth]{images/zpt/dimuon_pt_yes_zpt.pdf}}
            {\includegraphics[width=0.9\textwidth]{images/zpt/AN_2018_Njet1_zpt.png}}\\
            {\includegraphics[width=0.9\textwidth]{images/zpt/our_2018_Njet1_zpt.pdf}}
        \end{column}
    \end{columns}
\end{frame}

% \begin{frame}
%     \frametitle{Projection of $m_{\mu\mu}$ resolution impact on significance}
%     \begin{itemize}
%     \item Estimate necessary improvement in $m_{\mu\mu}$ resolution to obtain 5\% significance improvement in ggH channel.
%     \item Artificially improved $m_{\mu\mu}$ resolution of signal fit by 20\% and obtain a new expected significance for ggH channel.
%     \item Simple linear projection is made with current results to estimate the necessary improvement in $m_{\mu\mu}$ resolution to meet target results.
%     \end{itemize}
%     \begin{table}[h]
%     \centering
%     \begin{tabular}{|l|l|}
%         \hline
%         \textbf{Metric} & \textbf{Value} \\
%         \hline
%         Current expected significance of ggH channel  & 1.70 $\sigma$ \\
%         Expected significance with 20\% improvement in $m_{\mu\mu}$ res  & 1.93 $\sigma$ \\
%         $\Delta$ significance per percent (gradient) & 0.0116 $\frac{\sigma}{\%}$\\
%         5\% $\Delta$ significance from HIG-19-006 & 0.0776 $\sigma$  \\
%         Improvement in $m_{\mu\mu}$ res needed for 5\% $\sigma$ improvement & 6.70 \% \\
%         \hline
%     \end{tabular}
%     \label{tab:significance_improvement}
% \end{table}

% \end{frame}


\begin{frame}
    \frametitle{Summary}
\begin{itemize}
\item So far with the list of improvements, we obtain preliminary results that indicate improvement just in ggH channel as shown in table below compared to HIG-19-006.
\end{itemize}
\begin{table}[h!]
\centering
\begin{tabular}{|c|cc|cc|}
\hline
\textbf{Channel} & \multicolumn{2}{c|}{\textbf{HIG-19-006}} & \multicolumn{2}{c|}{\textbf{Our current results}} \\
 & Expected & Observed & Expected & Observed \\
\hline
ggH & 1.57 & 1.23 & 1.70 & - \\
\hline
\end{tabular}
\label{tab:ggh_comparison}
\end{table}
    \begin{itemize}
    \item Based on preliminary studies, we are confident in 5\% improvement target for Run~2.
    
    \begin{itemize}
    \item We aim to apply the improved methods to 2022 and 2023 samples as well.
    \end{itemize}
    \item We are happy to give a more comprehensive presentation on all/select topics mentioned above in a future meeting.
    \item More details can be found on AN-2023/017.
    \end{itemize}
\end{frame}


\begin{frame}
    \frametitle{Further Studies}
    In addition to improvements in $m_{\mu\mu}$ resolution, we list few topics we're interested in improving in the future.
    \begin{itemize}
    \item We are working on better channel categorization between ggH and VBF.
    \item Improve our current MVA to be systematics-aware/agnostic.
    \item Study switching to MiNNLO generated DY samples in Higgs mass region.
    \item Possible further improvement on $m_{\mu\mu}$ resolution by switching to new method used by the W Mass measurement analysis.
    \end{itemize}
\end{frame}


\end{document}
